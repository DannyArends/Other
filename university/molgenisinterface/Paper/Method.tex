\section{Method}
\subsection{Datastructuring and storage}
In R we adopted the R/qtl cross object format, however when sharing results with others 
(publication, etc) this binary format is not usefull. Here databases can help, storing huge amounts of data like publicly available 
and / or private datasets. Adding data to a database in a transaction based way ensures data consistency. 
Also database systems allow for third party software to store and retrieve data from the database directly or via
a supplied application programming interface (API). The XGAP \cite{morris07} (Xtensible Genotype and Phenotype) datamodel is used 
to format data in a homogenious matter when storing data into a molgenisDB. This XGAP format ensures data 
is extensible and consistent. The molgenisDB system ensures dataconsistency and -coherency, plus allowing easy 
access from R / JAVA / SOAP or by using HTML scripting.
\subsection{HPC of QTLs using molgenis}
A plugin was created to enable users to use the r/qtl package from the molgenis webenviroment. To use this plugin another machine is required 
to distribute jobs and do qtl analysis. This was done to separate qtl calculation from normal server operations.
On this website public datasets are available and the user can use the HPC cluster of the Rijks Universiteit of Groningen to do highperformance qtl analysis.
Also we created an overview of the current infrastructure needed for a functional plugin.