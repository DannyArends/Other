\section{Introduction}
\subsection{r/qtl}
R/qtl is an open source package for mapping QTL in experimental crosses for the R programming language\cite{broman09}\cite{broman03}.
It can be used to analyse several types of experimental crosses (Backcross (BC), F2, Recombinant inbred lines (RIL), Intercrosses (IC)) 
. Furthermore it offers many methods for QTL mapping (Composite interval mapping (CIM), expectation maximization (EM), marker regression (MR) and 
Extended Haley-Knot (eHK)). Because R/qtl supports different experimental crosses and contains many extra functions (plotting, genotype checking, etc).
Because it is opensource software no cost are associated with usage of the R-package (or the R-software infrastructure). Also its opensource nature
 makes it easy for others to contribute to the package. 
Our goal is to make complex QTL mapping methods widely accessible and allow users to focus on modeling rather than computing. R/qtl 
is becoming an increasingly popular environment for QTL analysis and has its own stabile userbase\cite{broman03}.
\subsection{Molgenis \& XGAP}
Storing information is very important, it becomes even more crucial when large -omics datasets are involved. These -omics sets are ussually generated 
in plain text format and read into a database system (mySQL/Oracle/msSQL). However these systems are not tailored to the needs of this biological data.
Providing ussually only storage for the most basic of datatypes (integers,strings,etc), they ussually lack biological dataformats, but are very good 
at storing data within a strict format. The molgenis database system using the extensible genotype \& phenotype (XGAP) datamodel aims to solve both these problem.
It does so by providing an easy to change datamodel (XGAP), with a generator to change the currentdatabase layout with a push of the button. The adaptability
of the system allows for easy managment of constantly changing biological data. This is furthermore enhanched by allowing for 3th party plugin to be 
connected to the database system. Molgenis was used in our project as out dataproviding software and a plugin was designed and implemented to enable 
molgenis users to perform R/QTL analysis from a webenviroment.
\subsection{High Performance Computing (HPC)}
The university of Groningen has an institute dedicated to high performance computing. The cluster is comprised of 168 machines running on a dualcore CPU. To 
facilitate high performance computations a PBS\cite{PBS2000} jobsceduler is present on the cluster, together with an installation of R-2.9.0. When computational job become too big for a desktop computer, these HPC facilities can be used. 
Another part of the project was focused on using the high speed computing facilities available in combination with the molgenis database system. 