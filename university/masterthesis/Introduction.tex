\newpage
\section{Introduction}
Genetic linkage analysis combined with the discovery of microarrays led to the first mapping of gene expression to an genomic location (quantative trait loci or eQTL) published in 2002 \cite{budding02}. Since this first mapping of eQTLs, other kind of biological traits have also been used for
QTL analysis for example: transcriptomics (eQTL), proteomics (pQTL) and metabolomics (mQTL)\cite{eQTL1}\cite{eQTL2}\cite{eQTL3}. By enabling researchers to map different quantitive traits
back to genomic location involved in regulation of these traits, QTL analysis has proven to be an invaluable tool to unravel 
heritable traits. Different QTL mapping strategies have evolved (or are under development). Each strategy has its own pros and cons, so there is no 
single best or universal QTL method to use. For example distributions of microarray expression data are different from LCMS data, and should be analysed 
using a different approach. The prefered approach depends on the statistical properties of the trait under consideration. 
In this paper we present several adaptations to the R/qtl package. These adaptations include:
Parallellization of many endophenotypes, visualizations of many endophenotypes analysis, Parallellization of bootstrap analysis and the addition 
of an opensource implementation of the MQM algorithm\cite{jansen93}. Furthermore for each function helpfiles and examples were created. 
\section{Background}
\subsection{R/qtl}
R/qtl is an open source package for mapping QTL in experimental crosses for the R programming language\cite{broman09}\cite{broman03}.
It can be used to analyse several types of experimental crosses (Backcross (BC), F2, Recombinant inbred lines (RIL), Intercrosses (IC)) 
. Furthermore it offers many methods for QTL mapping (Composite interval mapping (CIM), expectation maximization (EM), marker regression (MR) and 
Extended Haley-Knot (eHK)). Because R/qtl supports different experimental crosses and contains many extra functions (plotting, genotype checking, etc).
Because it is opensource software no costs are associated with usage of the R-package (or the R-software infrastructure). Also its opensource nature
 makes it easy for others to contribute to the package. 
Our goal is to make complex QTL mapping methods widely accessible and allow users to focus on modeling rather than computing. R/qtl 
is becoming an increasingly popular environment for QTL analysis and has its own stable userbase\cite{broman03}.
\subsubsection*{Current limitations of R/qtl}
\begin{itemize}
\item R/qtl has been developed for analysis of classical phenotypes and is therefore not ready for analysis of today�s 10,000-100,000s of molecular phenotypes.
\item R/qtl it has no features to make use of current day computerclusters or computers with multiple cores. 
\item It offers two main mapping methods for QTL mapping: \\1)a per marker approach using different methods and models, and\\ 2) CIM (composite interval mapping).
It does not yet have an implementation of MQM (multiple-QTL-mapping \cite{jansen93}\cite{jansen94}) 
\item R/qtl stores results from analysis in plain text format, it currently offers no support for databasing of raw and processed data 
or results from analyses.
\end{itemize}
\subsection*{Multiple QTL Mapping (MQM)}
MQM is a mapping method developed by R. C. Jansen \cite{jansen93}\cite{jansen94}. A commercial version (closed source) of the multiple QTL mapping
algorithm is implemented in MapQTL\cite{mapQTL}. MQM has some advantages above other QTL mapping methods: It reduces linkage by considering cofactors
to obtain a higher power when mapping QTLs, and it implements a backward model selection procedure using an analysis of deviance approach.
Because MQM is a more powerful algorithm for mapping QTLs it is computationally more intensive, and will thus require more time per analysis.
An opensource implementation of the MQM algorithm was also added to  R/qtl
\subsection{Molgenis \& XGAP}
Storing information is very important, it becomes even more crucial when large -omics datasets are involved. These -omics sets are usually generated 
in plain text format and read into a database system (mySQL/Oracle/msSQL). However these systems are not tailored to the needs of this biological data.
Providing only storage for the most basic of datatypes (integers,strings,etc), they usually lack biological dataformats, but are very good 
at storing data within a strict format. The Molgenis database system using the eXtensible Genotype \& Phenotype (XGAP) datamodel aims to solve both these problems.
It does so by providing an easy to change datamodel (XGAP), with a generator to change the current database layout with the push of a button. The adaptability
of the system allows for easy managment of constantly changing biological data. This is furthermore enhanced by allowing for 3rd party plugin to be 
connected to the database system. Molgenis was used in our project as out dataproviding software and a plugin was designed and implemented to enable 
Molgenis users to perform R/qtl analysis from a webenvironment.
\subsection{High Performance Computing}
The State university of Groningen has an institute dedicated to high performance computing. The cluster is comprised of 168 machines running on a dualcore CPU. To 
facilitate high performance computations a PBS\cite{PBS2000} jobscheduler is present on the cluster, together with an installation of R-2.9.0. When analysis of a QTLexperiment becomes too big for a desktop computer, these HPC facilities can be used. 
Another part of the project was focused on using the high performance computing facilities available in combination with the Molgenis database system. 