\section*{Abstract}
This thesis was created as the result of six months research at the department of bioinformatics at the state university of Groningen.
The research during this period focused on (1) Multiple QTL mapping and (2) Solutions for repeated analysis of quantative traits,
these traits are ussually generated using high throughput molecular techniques. Analysis of these large datasets is becoming a challenge with
modern day computers and statistics.
\\\\
(1) Multiple QTL mapping (MQM) - MQM is a modeling strategy for QTL mapping, the algorithm (backward) selects cofactors trying to maximize the fit of the 
model to the quantative trait. Dissecting a trait into multiple QTLs the fit of the QTLmodel with X+1 cofactors will always be better than a model with X cofactors, 
to compensate for this effect the algorithm has to penalize extra cofactors in the model. An optimal fit of the model is determined using the AIC to maximize 
the amount of variance explained by genetic components.
\\\\
(2) Repeated analysis - This repetition is due to the advent of -omics datasets generated by highthroughput molecular methods. High throughput
 enables researchers to get more endophenotypes then ever before. The mapping of these quantative traits back to the genome, and the discovery of
regulation involved, is computational very intensive. For single trait analysis this can be performed on a standard desktop computer.
This is however not the case when mapping hundreds of thousands of traits obtained from large scale genomewide association studies (GWAS), or large scale linkage analysis.
\\\\
The result of this research is the adaptation of R/qtl, a commonly used package for QTL analysis on experimental populations.
  \\- Addition of Multiple QTL mapping to R/qtl
  \\- Multicore QTL mapping using SNOW or HPC/V cluster
  \\- Multicore QTL permutation using SNOW
  \\- R/qtl analysisplugin for the XGAP system
\\\\\\
I hope you will enjoy reading this thesis about the mapping of multiple quantative traits using multiple QTL models,
\\
Danny Arends
\\
Masterstudent State university of Groningen