\newpage
\section{Discussion}

\subsection{Speeding up QTL \& Permutation analysis}
To calculate even more QTL profiles or the same amount in less time several adaptations could be considered in the future:
\\- Not mapping for QTLs when modelselection fails to fit a model for a single trait
\\- Using the GPU (in addition to the CPU) which has multiple cores and faster memory
\\- Gradual permutation strategy - when we find a peak significant after X iterations stop calculating.
\subsection{Multiple QTL mapping}
\subsubsection*{Why and When use the multiple QTL mapping (MQM) algorithm?}
One of the main advantages of MQM over currently available methods is the higher precision when performing QTL analysis.
Also using co-factors and doing backward model selection can help identify previously unknown locations underlying
complex traits. To summarize the advantages of MQM:
\begin{itemize}
\item Better resolution than using other QTL mapping methods
	\begin{itemize}
	\item More available degrees of freedom, thus enabling users to create larger QTL models or 
	consider more cofactors.
	\item Interval mapping to get narrower confidence intervals, and thus reduce the number of genes 
	(possible regulators of the trait) underlying the QTL 
	\item Possibility to discover negatively linked QTLs on the same chromosome. By using cofactors 
	different hypotheses are tested on different locations thus making it possible to unweave linkage between markers.
	\end{itemize}
\item Automatic backward model selection
\end{itemize}
However, there are also situations in which other QTL mapping methods present in  R/qtl are favoured:
\begin{itemize}
\item Ultra dense markermaps, here resolution is not limited by the markerset, but by the number of recombinations events between individuals. 
This means that no extra resolution is obtained using MQM, so a markerbased approach would be computational less intensive. 
\item This also holds when using a huge number of individuals for QTL analysis ($>3000$). Altough MQM scales 
linearly based on the number of individuals when mapping QTLs, it is not possible to use a parallellization strategy because of 
dependencies between individuals.
\end{itemize}
\subsubsection*{Model selection}
Some limitations to modelselection exist in the current implementation of the MQM algorithm. We can't consider all markers as cofactors 
when doing backward elimination because of limited power for statistical analysis. More power can be obtained by including more individuals, 
in general we can set as many cofactors as we have individuals. If we want to test all markers as cofactors one should do multiple MQM runs with
different markers selected as cofactor, to assess every markers importance. Furthermore, once a marker is dropped it is removed from the final model and not again looked at for inclusion in a following step. This could lead to erroneously dropping of cofactors when different cofactors explain the same part of the trait variance.

\subsection{Benchmarking}
Benchmarking shows an almost linear improvement in analysis time when using more computers in a cluster 
(not counting data transfer times from and to the database). This linear scaling can also be obtained on a single desktop
machine with multiple cores. The second strategy using SNOW in combination with clustermachines however has somewhat 
more overhead because of starting more R instances. This methode will scale across a large number of cores. With an increasing number of cores result will be increasingly less than linear, but every machine will improve calculation time.
However using both cores of the HPC machines doubles the effective number of machines in the cluster from 168 to 336 machines. Thus 
overall we can calculate ~300 traits using the cluster in the time it takes a desktop pc to do a single profile.

\subsection{HPC of QTLs using the molgenisplugin}
Further extensions of the plugin could be made, a simple roadmap for the plugin:
\\- A priori checking of tables selected.
\\- Expose more scanning routines (CIM, MQM).
\\- Expose all available parameters (scanone, CIM, MQM).
\\- Better estimation of runtime on cluster.
\\- Adding the option to do concurrent runs on the cluster
\\- Better status overview using active polling of the calculationserver

\subsection{Visualizations}
Visualizations are used to tell a story. By adding more visualizations to  R/qtl we hope to expand the vocabulary of researchers.
Some visualizations that were added:
\begin{itemize}
\item Singletrait permutation plot - Shows 5\% and 10\% FDR but also in blue the distribution of scores during permutation plotted per marker.
\item Multitrait contour plot - Shows a contour plot where the outlines are different LOD levels for QTL.
\item Multitrait heatmap - Showing QTL profiles for multiple traits at once.
\item Cis/Trans plot - Showing QTL location versus trait location in 2D across the genome. This plot can be made when a user mapped a trait which also has a genetic location (gene expression, proteins)
\end{itemize}